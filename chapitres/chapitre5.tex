\documentclass[main.tex]{subfiles}

\chapter{Mise en application dans des vrais matchs}

Maintenant, que nous avons notre équipe et que nous avons confirmé qu'il n'est pas possible pour le moment de réaliser une IA qui jouera cette équipe, nous allons avec des humains de niveaux différents tester ces équipes dans les conditions réelles.

\section{Pokémon Showdown}

Les VGC ayant déjà commencé, il ne nous est pas possible d'y participer afin de déterminer la valeur concrète de nos équipes face à des joueurs réelles et dans des conditions plus favorables. Il est en revanche possible de simuler des combats dans un environnement qui nous permettrait de mettre nos équipes en situation réelle.

Cette environnement s'appelle Pokémon Showdown.

Pokémon Showdown est un simulateur de combat Pokémon sur internet. Il propose un système de rencontres aléatoires dans un format souhaité, et il est aussi possible de se battre contre un adversaire précis.

\subsection{Les spécimens}

Nous avons donc trouver des spécimens qui vont aller utiliser ses équipes dans ce simulateur et qui iront jouer contre des joueurs réels mais inconnu de niveaux différent. Nous enregistrerons le plus de combat possible selon leurs disponibilités et analyseront les résultats.

Nous définirons trois classes de spécimens :

\begin{itemize}
    
    \item Débutant
    
    Un débutant est un joueur n'ayant pas ou peu de connaissances dans Pokémon. Il n'aura aucune ou alors que de vagues idées sur comment son équipe fonctionne.
    
    \item Intermédiaire
    
    Un joueur intermédiaire ayant des Pokémon relatives à Pokémon et au bon déroulement d'un combat. Il connaît son équipe et sait ce qu'ils font. Il n'est cependant pas efficient dans sa prise de décision.
    
    \item Avancé
    
    Un joueur avancé est un joueur qui connaît ces principes de méta et qui sait les appliquer en combat. Il sait s'adapter, et prendre une décision correcte.
    
\end{itemize}

\section{Évaluation des résultats}

\subsection{Débutant}

\subsection{Intermédiaire}

\subsection{Avancé}