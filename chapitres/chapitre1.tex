\documentclass[main.tex]{subfiles}

\chapter{Analyse des solutions existantes}

\section{Deep Blue}

\section{AlphaGo}

AlphaGo est un programme informatique capable de jouer au jeu de Go\footnote{jeu de stratégie au tour par tour traditionnellement chinois}. En octobre 2015, il devient le premier programme à battre le joueur professionnel français Fen Hui, en mars 2016, il bat l'un des meilleurs joueurs mondiales Lee Sedol et enfin, en mai 2017 il bat le champion du monde, Ke Jie, avant d'être mis à la retraite.

Il a été développé par l'entreprise Britannique DeepMind Technologies. Son algorithme combine des techniques de parcours de graphe et d'apprentissage automatique, à partir de bataille contre d'autres humains, d'autres ordinateurs mais surtout contre lui-même.

En octobre 2017, son algorithme est amélioré dans la version AlphaGo Zero qui atteint un niveau supérieur en jouant uniquement contre lui-même et en décembre 2017 devient capable de battre tout les joueurs et ordinateurs au go, mais aussi aux échecs et au shogi, et ce, toujours par auto-apprentissage.

Les premières versions d'AlphaGo ont été réalisé avec l'utilisation de la méthode de Monte-Carlo, guidé par un réseau de valeur et un réseau d'objectifs, implémentés en utilisant un réseau de neuronne profond. Il a été entraîné pour imiter le joueur humain, en retrouvant la réponse aux coups dans toutes les parties qu'il a enregistrés. Passé un certain niveau, il s'est entraîné contre lui-même utilisant l'apprentissage par renforcement pour s'améliorer.

En revanche, dans une nouvelle étude, par Nature, DeepMin révèle que la version AlphaGo Zero utilise une architecture plus simple, n'utilise plus la méthode de Monte-Carlo, ni de connaissances humaines mais parvient tout de même à atteindre un meilleur niveau que ses versions précédentes.

\subsection{La méthode de Monte-Carlo}

Blabla\footnote{https://www.chemie.unibas.ch/\~meuwly/download/ulam.mc.pdf}

La méthode Monte-Carlo désigne une famille de méthode algorithmiques visant à calculer une valeur numérique approchée en utilisant des procédés aléatoires.