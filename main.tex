\documentclass[12pt, twoside, openright]{report}

%----------------------------------------------------------------------------------------
%	PACKAGES
%----------------------------------------------------------------------------------------

\usepackage{emptypage}
\usepackage{geometry}
\usepackage[utf8x]{inputenc}
\usepackage[french]{babel}
\usepackage[T1]{fontenc}
\usepackage{amsmath}
\usepackage{amsfonts}
\usepackage{amssymb}
\usepackage{graphicx}
\usepackage{subfiles}
\usepackage{fullpage}
\usepackage{fancyhdr}
\usepackage{shorttoc}
\usepackage{fontspec}
\usepackage{xltxtra}
\usepackage{xcolor}
\usepackage{sectsty}
\usepackage[Lenny]{fncychap}
\usepackage[backend=biber,style=numeric,]{biblatex}

%----------------------------------------------------------------------------------------
%	BIBLIOGRAPHIE
%----------------------------------------------------------------------------------------

\addbibresource{bibliographie/bibliographie.bib}

%----------------------------------------------------------------------------------------
%	STYLES
%----------------------------------------------------------------------------------------

\ChNumVar{\fontsize{60}{62}\usefont{OT1}{ptm}{m}{n}\selectfont\textcolor{red}}

\setmainfont[Mapping=tex-text]{Lato}

\geometry{
    paper=a4paper,
    inner=3cm,
    outer=2.5cm,
    top=2.5cm,
    bottom=3.5cm
}

\pagestyle{fancy}
\usepackage{etoolbox}
\patchcmd{\chapter}{\thispagestyle{plain}}{\thispagestyle{fancy}}{}{}
\renewcommand\headrulewidth{0pt}
\fancyhead[L]{}
\fancyhead[C]{}
\fancyhead[R]{}

\makeatletter
\patchcmd{\@makechapterhead}{\vspace*{50\p@}}{\vspace*{-35\p@}}{}{}
\patchcmd{\@makeschapterhead}{\vspace*{50\p@}}{\vspace*{-35\p@}}{}{}
\patchcmd{\DOTI}{\vskip 80\p@}{\vskip 40\p@}{}{}
\patchcmd{\DOTIS}{\vskip 40\p@}{\vskip 0\p@}{}{}
\makeatother

\renewcommand\thesection{\color{red}\thechapter.\arabic{section}}

\newenvironment{acknowledgements} {\renewcommand\abstractname{Remerciements}\begin{abstract}} {\end{abstract}}


\setcounter{tocdepth}{3}
\setcounter{secnumdepth}{3}


%----------------------------------------------------------------------------------------
%	INFORMATIONS
%----------------------------------------------------------------------------------------

\author{Bastian SZCZYGIELSKI}
\title{Mémoire de fin d'études}

\begin{document}

%\subfile{cover/page_garde}
%----------------------------------------------------------------------------------------
%	TITLE PAGE
%----------------------------------------------------------------------------------------

\setlength{\parindent}{0cm}
\setlength{\parskip}{1ex plus 0.5ex minus 0.2ex}
\newcommand{\hsp}{\hspace{20pt}}
\newcommand{\HRule}{\rule{\linewidth}{0.5mm}}

\begin{titlepage}
  \begin{sffamily}
  \begin{center}

    \textsc{\LARGE MASTER MIAGE 2ème année \linebreak Université Paris Nanterre}\\[2cm]

    \textsc{\Large Mémoire de fin d'études}\\[1.5cm]

    \HRule \\[0.4cm]
    { \huge \bfseries Générer la meilleure équipe Pokémon pour les VGC 2019\\[0.4cm] }

    \HRule \\[2cm]
    \includegraphics[scale=0.40]{img/logo_nanterre.jpg}
    \hspace{2cm}
    
    \vfill
  \begin{minipage}{0.4\textwidth}
      \begin{flushleft} \large
        \emph{Auteur :}\\ \textsc{Bastian SZCZYGIELSKI}\\
      \end{flushleft}
    \end{minipage}
    \begin{minipage}{0.4\textwidth}
      \begin{flushright} \large
        \emph{Tuteur :}\\ \textsc{Sana BEN HAMIDA}\\
      \end{flushright}
    \end{minipage}
    \vfill
    {\large Mars 2019 — Juillet 2019}
  \end{center}
  \end{sffamily}
\end{titlepage}

\leavevmode\thispagestyle{empty}\newpage

%----------------------------------------------------------------------------------------
%	Remerciement
%----------------------------------------------------------------------------------------

\begin{acknowledgements}

Merci maman, merci papa, merci Nintendo et merci la dépression. Vous m'avez porté jusqu'à la meilleure version de moi-même. Sans vous, je ne serai pas le gamer blazé que je suis aujourd'hui et qui aura trouvé la motivation pour terminer ses études.

Merci M Delbot, d'avoir partagé ce projet avec moi, qui m'a actuellement donné envie de m'adonner à la tâche de réaliser un mémoire et de m'avoir accompagné dans ce processus.

Merci à tout ceux qui ont aidé à la réalisation de ce mémoire. Ce mémoire serait vide sans les chances et les avis qui m'ont été donné.

Mention spécial à mon camarade Ludwig, joueur professionel de Pokémon, 5ème dan de Pokémon Tōnamento, qui m'a fourni toutes les informations les plus critiques concernant Pokémon.

\end{acknowledgements}

\leavevmode\thispagestyle{empty}\newpage

%----------------------------------------------------------------------------------------
%	RESUME
%----------------------------------------------------------------------------------------

\begin{abstract}
	Dans ce mémoire, nous allons parler de la possibilité d'automatiser la génération d'équipe ainsi que la gestion d'un combat dans le cadre des formats VGC.
	
	Nous allons présenter tous les parties en lien avec ce travail afin de bien mettre en contexte les éléments qui vont intervenir dans le cours de cette réalisation.
	
	Nous allons voir et étudier les possibilités qui s'offrent à nous dans le monde de l'intelligence artificielle pour résoudre notre problème.
	
	Nous allons chercher à trouver une solution à partir des études réalisés et des contraintes imposées ou supposées.
	
	Enfin, nous allons appliquer la solution que nous avons trouvé suite à notre recherche et analyser les résultats obtenus.
\end{abstract}

\leavevmode\thispagestyle{empty}\newpage

%----------------------------------------------------------------------------------------
%	Pre Face
%----------------------------------------------------------------------------------------

\section*{Motivations}

Les défis, l'espoir et l'amitié.

C'est mon destin.

Ça demande du courage mais c'est un voyage d'apprentissage.

\section*{Objectifs}

Être le meilleur dresseur.

Tout faire pour être vainqueur.

Parcourir la terre entière.

Traquer les secrets et mystères des Pokémon et les attraper tous.

Être le plus grand maître Pokémon.

\leavevmode\thispagestyle{empty}\newpage

%----------------------------------------------------------------------------------------
%	SOMMAIRE
%----------------------------------------------------------------------------------------

\shorttoc{Sommaire}{1}

%----------------------------------------------------------------------------------------
%	INTRODUCTION
%----------------------------------------------------------------------------------------

\subfile{introduction}

%----------------------------------------------------------------------------------------
%	CHAPITRES
%----------------------------------------------------------------------------------------

\subfile{chapitres/chapitre1}

\subfile{chapitres/chapitre2}

\subfile{chapitres/chapitre3}

\subfile{chapitres/chapitre4}

\subfile{chapitres/chapitre5}

%----------------------------------------------------------------------------------------
%	ANNEXES
%----------------------------------------------------------------------------------------

\subfile{annexes/annexes}

%----------------------------------------------------------------------------------------
%	BIBLIOGRAPHIE
%----------------------------------------------------------------------------------------

\printbibliography

\newpage

%----------------------------------------------------------------------------------------
%	TABLE DES MATIÈRES
%----------------------------------------------------------------------------------------

\tableofcontents

\listoffigures

\listoftables

\end{document}