\documentclass[memoire.tex]{subfiles}

\chapter*{Introduction}

\section*{Pokémon}

Pokémon est une franchise appartenant au groupe japonais Nintendo qui a vu le jour en 1996. Originellement une série de jeu vidéo ludique, elle a été exploité sous plusieurs formes tel qu'un anime\footnote{Dessin animé japonais}, des films, des manga\footnote{Bande dessiné japonaise} ou encore un jeu de cartes à collectionner.

Le nom Pokémon provient de la romanisation des mots "Pocket Monsters" ou "Monstres de poches" en japonais "Poketto Monsuta" (ポケットモンスター). Autre qu'étant le nom de la franchise, le mot désigne aussi les créatures qui sont au centre de la franchise.

Dans le monde fictifs des jeux Pokémon, le principe est de parcourir le monde virtuel et de vaincre et capturer des Pokémon, et de se battre avec d'autres propriétaires de Pokémon nommés "Dresseurs" et d'acquérir des récompenses symboliques après les avoir vaincu. Notamment, parmi les objectifs du jeu, les principales sont acquérir tous les Pokémon et vaincre la ligue Pokémon, un rassemblement des dresseurs les plus forts dans le jeu.

Les Pokémon se caractérisent par plusieurs traits individuelles et collectifs tels que des traits physiques et comportementales qui ont un effet sur leur valeur. Ils peuvent évoluer et prendre une nouvelle forme et ils peuvent apprendre des techniques qu'ils utiliseront en combat ou hors-combat.

Les combats se déroulent en tour par tour, ou chaque dresseur prend le temps de réaliser une action, qui peut être d'utiliser un objet dans son inventaire, prendre la fuir si possible, changer de Pokémon ou utiliser la capacité d'un Pokémon. Cette action est déterminé avant le début du tour et l'ordre des actions sont déterminés par la vitesse des Pokémon en combat.

Chaque jeu Pokémon définit de nouveaux objectifs en plus de l'objectif principale qui est de rassembler tous les Pokémon disponibles dans sa génération. Dans la première génération de jeu vidéo, les Versions Bleu, Rouge et Jaune, on commence simplement avec le but de vaincre les maîtres d'arènes de la région de Kanto afin d'accéder à la ligue Pokémon de cette région. Dans la troisième génération, les versions Rubis, Saphir et Émeraude, entre autre de vaincre la ligue, il rajoute des nouveaux objectifs comme gagner les concours Pokémon où le Pokémon est jugé selon la catégorie définie (Beauté, Force...), ou la tour de force, une extension de la ligue ou le joueur se voit affronter une série de dresseurs qui deviennent progressivement de plus en plus fort.

Chaque joueur a l'opportunité de définir ses propres objectifs dans les mesures du possibles et du réalisables.

\section*{L'eSport}

L'eSport, ou les sports électroniques désigne la pratique des jeux vidéos seul ou en équipe en compétition avec à la clé des récompenses diverses et variés\cite{ESPORT}. Ils connaissent une forte croissance dans les années 80 à la naissance des premiers jeux vidéos en ligne et de la possibilité de jouer avec et contre d'autres joueurs.

Ils existent plusieurs organisations dans le monde qui organisent des évènements eSport où les joueurs peuvent venir se rencontrer et s'affronter et chaque année de plus grands évènements se déroulent afin de déterminer qui est le meilleur dans leur domaine, comme on pourrait le voir au Roland Garros, à la coupe du monde de football ou aux jeux olympiques. Parmi ses organisations, on retrouve l'Evolution Championship Series (EVO), la Major League Gaming (MLG) ou encore les Video Game Championship (VGC) dans lesquelles on retrouve un tournoi de Pokémon.

C'est en 2008 que l'International eSport Federation (IeSF) est né en Corée du Sud. Il s'agit d'une organisation mondiale dont le but est de légitimiser l'eSport comme un sport à part entière, un sujet qui fait débat. La France intégrera cette organisation en 2018.

On trouve en France un très bon potentiel puisque le revenu généré par l'eSport en 2016 était de 22.4 millions de dollars américain avec une audience de 1 404 964 visiteurs uniques\cite{AFJ}.

En 2018, Sporsora rapporte\cite{SPOR} en France 5 066 000 de consommateur d'eSport, soit 12\% des internautes avec un impact économique de 30 millions de dollars américain. La France est alors le troisième plus gros consommateur d'eSport en Europe.

En avril 2016, le Comité International Olympique annonce reconnaître l'eSport comme un sport à part entière et discute de la possibilité des jeux vidéo aux jeux olympique\cite{AFK}. Peu après l'obtention par Paris des Jeux Olympiques en 2024, Tony Estanguet, membre du Comité International Olympique et coprésident du Comité de candidature Paris 2024 se dit prêt à discuter avec le CIO et les représentants du monde du sport électronique pour éventuellement y inclure la discipline\cite{PATATE}. Cependant, l'eSport étant reconnu comme un sport à part entière continue de faire controverse\cite{LC}.

\section*{Pokémon en tant qu'eSport}

De manière officiel, il existe pour le jeu vidéo la compétitions des VGC et pour le jeu de cartes à collectionner, il existe les TCG Championship. Ces compétitions se déroule sur le long de l'année pour pouvoir se qualifier aux championnats du monde qui leur sont dédiés nommé les Pokemon Worlds. Chaque année, pour y participer en tant que compétiteur, il faut participer à des tournois durant le cours de l'année pour accumuler un certain nombre de point.

En 2018, les Pokémon World Championship qui ont eu lieu du 24 au 26 août a eu un temps de visionnage total de 429 435 heures, pour un temps de diffusion total de 29 heures. Il a été regardé par 14 682 spectateurs unique en direct pour un total de 636 700 visionnages sur les chaînes de la plateforme twitch.tv\cite{ESC}.

Le tournoi en lui-même offrait 500 000 dollars américain divisé dans trois catégories d'âges et au travers de trois jeux différents\cite{COMPVG}.