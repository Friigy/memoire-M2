\documentclass[memoire.tex]{subfiles}

\chapter*{Introduction}

\section*{Pokémon}

Pokémon est une franchise appartenant au groupe japonais Nintendo qui a vu le jour en 1996. Originellement une série de jeu vidéo ludique, elle a été exploité sous plusieurs formes tel qu'un anime\footnote{Dessin animé japonais}, des films, des manga\footnote{Bande dessiné japonaise} ou encore un jeu de cartes à collectionner.

Le nom Pokémon provient de la romanisation des mots "Pocket Monsters" ou "Monstres de poches" en japonais "Poketto Monsuta" (ポケットモンスター). Autre qu'étant le nom de la franchise, le mot désigne aussi les créatures qui sont au centre de la franchise.

Dans le monde fictifs des jeux Pokémon, le principe est de parcourir le monde virtuel et de vaincre et capturer des Pokémon, et de se battre avec d'autres propriétaires de Pokémon nommés "Dresseurs" et d'acquérir des récompenses symboliques après les avoir vaincu. Notamment, parmi les objectifs du jeu, les principales sont acquérir tous les Pokémon et vaincre la ligue Pokémon, un rassemblement des dresseurs les plus forts dans le jeu.

Les Pokémon se caractérisent par plusieurs traits individuelles et collectifs tels que des traits physiques et comportementales qui ont un effet sur leur valeur. Ils peuvent évoluer et prendre une nouvelle forme et ils peuvent apprendre des techniques qu'ils utiliseront en combat ou hors-combat.

Chaque jeu Pokémon définit de nouveaux objectifs en plus de l'objectif principale qui est de rassembler tous les Pokémon disponibles dans sa génération. Dans la première génération de jeu vidéo, les Versions Bleu, Rouge et Jaune, on commence simplement avec le but de vaincre les maîtres d'arènes de la région de Kanto afin d'accéder à la ligue Pokémon de cette région. Dans la troisième génération, les versions Rubis, Saphir et Émeraude, entre autre de vaincre la ligue, il rajoute des nouveaux objectifs comme gagner les concours Pokémon où le Pokémon est jugé selon la catégorie définie (Beauté, Force...), ou la tour de force, une extension de la ligue ou le joueur se voit affronter une série de dresseurs qui deviennent progressivement de plus en plus fort.

Chaque joueur a l'opportunité de définir ses propres objectifs dans les mesures du possibles et du réalisables.

https://fr.wikipedia.org/wiki/Pok%C3%A9mon

\section*{L'eSport}

L'eSport, ou les sports électroniques désigne la pratique des jeux vidéos seul ou en équipe en compétition avec à la clé des récompenses diverses et variés. Ils connaissent une forte croissance dans les années 80 à la naissance des premiers jeux vidéos en ligne et de la possibilité de jouer avec et contre d'autres joueurs.

Ils existent plusieurs organisations dans le monde qui organisent des évenements eSport où les joueurs peuvent venir se rencontrer et s'affronter et chaque année de plus grands évenements se déroulent afin de déterminer qui est le meilleur dans leur domaine, comme on pourrait le voir au Roland Garros, à la coupe du monde de football ou aux jeux olympiques. Parmi ses organisations, on retrouve l'Evolution Championship Series (EVO), la Major League Gaming (MLG) ou encore les Video Game Championship (VGC) dans lesquelles on retrouve un tournoi de Pokémon.

C'est en 2008 que l'International eSport Federation (IeSF) est né en Corée du Sud. Il s'agit d'une organisation mondiale dont le but est de légitimiser l'eSport comme un sport à part entière, un sujet qui fait débat\footnote{https://www.la-croix.com/Sport/Ce-sport-derange-monde-sportif-2017-05-12-1200846649}. La france intégrera cette organisation en 2018.

On trouve en France un très bon potentiel puisque le revenu généré par l'eSport en 2016 était de 22.4 millions de dollars américain\footnote{https://www.afjv.com/news/6153_paypal-devoile-les-derniers-chiffres-du-marche-de-l-e-sport.htm}


https://fr.wikipedia.org/wiki/Sport_%C3%A9lectronique
https://www.pokepedia.fr/VGC_2018
https://www.afjv.com/news/9675_le-monde-de-l-esport-decrypte-par-sporsora-et-des-experts.htm

\section*{Pokémon en tant qu'eSport}